%%%%%%%%%%%%%%%%%%%%%%%%%%%%%%%%%%%%%%%%%%%%%%%%%%

%% Every LaTeX document has a document class. For this document, I will use a custom resume class.

\documentclass[11pt]{resume}

%%%%%%%%%%%%%%%%%%%%%%%%%%%%%%%%%%%%%%%%%%%%%%%%%%

%% Next, I’ll provide some semantic information. Each of these commands is defined in the résumé class. 

\name{Sam}{Boosalis}
\address{San Francisco}
\mobile{+1 (617) 922-0791}
\email{samboosalis@gmail.com}
\homepage{http://sboosali.github.io}

%%%%%%%%%%%%%%%%%%%%%%%%%%%%%%%%%%%%%%%%%%%%%%%%%%

%% This begins the document. The command makecvheader is from resume.cls and will setup the basic template.

\begin{document}

\makecvheader[C]

%%%%%%%%%%%%%%%%%%%%%%%%%%%%%%%%%%%%%%%%%%%%%%%%%%

  %% This LaTeX class defines =\cvsection= to separate sections of the résumé by a
  %% horizontal line and some blank space. The only argument for =\cvsection= sets
  %% the contents of the section header.

\cvsection{Education}

\begin{cventries}

  %% Each =cvsection= is composed of multiple =cventries=.

%%%%%%%%%%%%%%%%%%%%%%%%%%%%%%%%%%%%%%%%%%%%%%%%%%

\cventry
  {University of Kansas}
  {Lawrence, KS, USA}
  {B.S. in Computer Science, 3.4 GPA}
  {Aug. 2015 – Exp. Dec. 2018}
  {
    \begin{cvitems}
    \item { Participated in ACM and hackathon competitions }
    \item { Coursework includes Software Engineering, Programming Languages,
        and Communication Networks }
    \end{cvitems}
  }

  %% =cventry= can be though of as a function that takes a variable 5 arguments.
  %% The first will be the heading used. The second will be on the right-hand side
  %% in /italics/. The third will be in /italics/ immediately below the first. The
  %% fourth will be on the right-hand side of the third argument. The last argument
  %% provides items for the entry.

  %% Visually:
  %% 
  %% | left         | right        |
  %% |--------------+--------------|
  %% | argument 1   | /argument 2/ |
  %% | /argument 3/ | /argument 4/ |

  %% Semantically:
  %% 
  %% | left                  | right                   |
  %% |-----------------------+-------------------------|
  %% | Company or University | /Location/              |
  %% | /Position or degree/  | /Start date - end date/ |

\end{cventries}

%%%%%%%%%%%%%%%%%%%%%%%%%%%%%%%%%%%%%%%%%%%%%%%%%%

\cvsection{Work Experience}

\begin{cventries}

%%%%%%%%%%%%%%%%%%%%%%%%%%%%%%

\cventry
  {Amazon.com, Inc.}
  {Seattle, WA, USA}
  {SDE Intern}
  {Summer 2017}
  {
    \begin{cvitems}
    \item { Worked on Mobile Identity team which manages the login screens for
            Amazon apps }
    \item { Project made it easier for teams to register new devices through
            Identity Services }
    \item { Used Agile development principles in design and development of
            project }
    \end{cvitems}
  }

%%%%%%%%%%%%%%%%%%%%%%%%%%%%%%

\cventry
  {Lexmark Enterprise Software}
  {Lenexa, KS, USA}
  {Software Engineer Intern}
  {Summer 2015, Summer 2016}
  {
    \begin{cvitems}
    \item { Worked on the Client Architecture team which builds the JavaScript
        web framework which other teams use to build enterprise solutions }
    \item { Participated in high level design decision conversations }
    \item { Project moved the web framework away from in-house solutions to
        better maintained open source projects while preserving legacy
        compatibility }
    \item { Asked to return after impressive first year }
    \end{cvitems}
  }

%%%%%%%%%%%%%%%%%%%%%%%%%%%%%%

\cventry
  {Together+Clinic}
  {Lincoln, NE, USA}
  {Design Studio Intern}
  {Spring 2015}
  {
    \begin{cvitems}
    \item { Startup building web app to let doctors track patients recovering
        from surgery without frequent checkup visits }
    \item { The web interface is used by both patients to record progress and
        doctors to track progress }
    \item { Team used Scrum development principles for quick response and user
        focused design }
    \end{cvitems}
  }

\end{cventries}

%%%%%%%%%%%%%%%%%%%%%%%%%%%%%%%%%%%%%%%%%%%%%%%%%%

\cvsection{Honors \& Awards}

\begin{cvhonors}

  \cvhonor
  {3rd Place}
  {JayHacks Hackathon}
  {Lawrence, KS, USA}
  {2017}

  \cvhonor
  {Grand Prize}
  {Google Code-in}
  {Mountain View, CA, USA}
  {2013}

\end{cvhonors}

 %% The =makecvfooter= command gives a nice footer that will be put at the bottom
 %% of each page. This can give us the document title and page numbering. In
 %% addition, the LastPage command will tell us how many pages there are in case
 %% we misplace a page while printing.

%%%%%%%%%%%%%%%%%%%%%%%%%%%%%%%%%%%%%%%%%%%%%%%%%%

\makecvfooter
    {BAUER}
    {\thepage}
    {\pageref{LastPage}}

%%%%%%%%%%%%%%%%%%%%%%%%%%%%%%%%%%%%%%%%%%%%%%%%%%

\end{document}

%%%%%%%%%%%%%%%%%%%%%%%%%%%%%%%%%%%%%%%%%%%%%%%%%%